% $Header: /cvsroot/latex-beamer/latex-beamer/solutions/conference-talks/conference-ornate-20min.en.tex,v 1.6 2004/10/07 20:53:08 tantau Exp $

%to make a handout
%documentclass[11pt]{article}
%\usepackage{beamerarticle}

%to suppress all frame titles in the article mode
%\setbeamertemplate<article>{frametitle}{}
%\only<article>{This text is shown only in article mode.}
\documentclass[noamsthm]{beamer}

\usepackage{tipa}
\let\ipa\textipa
%\usepackage{pgf}
\usepackage{graphicx}
%Phonology
	%to create a vowel space
	\usepackage{vowel}

\mode<presentation>
{
%\usetheme{Malmoe}
%\usetheme{Warsaw}

%\usetheme{Montpellier}
%\usetheme{AntibesGina}

%shows full sections
%\usetheme{PaloAlto}
\usetheme{Hannover}

%lots of info on the bottom
%\usetheme{Madrid}

%no sections shown
%\usetheme{Rochester}
%\usetheme{Pittsburgh}

%very different
%\usetheme{Bergen}

  % or ...

 %\setbeamercovered{transparent}
  % or whatever (possibly just delete it)
}


\usepackage[english]{babel}
% or whatever

%\usepackage[latin1]{inputenc}
% or whatever

%\usepackage{times}
%\usepackage[T1]{fontenc}
% Or whatever. Note that the encoding and the font should match. If T1
% does not look nice, try deleting the line with the fontenc.

\usepackage{covington}

\title[Beamer Examples] % (optional, use only with long paper titles)
{Some Beamer Examples}

%\subtitle
%{Include Only If Paper Has a Subtitle}

\author[Author1,Author2] % (optional, use only with lots of authors)
{Author One  \and Author Two }
% - Give the names in the same order as the appear in the paper.
% - Use the \inst{?} command only if the authors have different
%   affiliation.

\institute[Concordia] % (optional, but mostly needed)
{Concordia University}
% - Use the \inst command only if there are several affiliations.
% - Keep it simple, no one is interested in your street address.

\date[LaTeX 2008] % (optional, should be abbreviation of conference name)
{LaTeX Workshop, 2008}

% Delete this, if you do not want the table of contents to pop up at
% the beginning of each subsection:
% \AtBeginSubsection[]
% {
%   \begin{frame}<beamer>
%     \frametitle{Outline}
%     \tableofcontents[currentsection,currentsubsection]
%   \end{frame}
% }


% If you wish to uncover everything in a step-wise fashion, uncomment
% the following command: 

%\beamerdefaultoverlayspecification{<+->}


\begin{document}

\begin{frame}
  \titlepage
\end{frame}

\begin{frame}
  \frametitle{Outline}
  \tableofcontents
  % You might wish to add the option [pausesections]
\end{frame}

\section{Lists}

\subsection{Enumerated Lists}

\begin{frame}
\frametitle{Enumerated Lists}


Here is the automated way enumerated lists look

\begin{enumerate}
	\item This is the first level in the first level
	\item This is the second item in the first level
	\begin{enumerate}
	\item This is the first item in the second level
	\item This is the second item in the second level
	
\end{enumerate}
	\item This is the third item in the first level
	\item This is the fourth item in the first level
\end{enumerate}

\end{frame}

\subsection{Itemized Lists}

\begin{frame}
\frametitle{Itemized Lists}


\begin{itemize}
	\item here is a bunch of embedded items
	\item buy groceries
	\begin{itemize}
		\item potatoes
		\item celery
		\item frying chicken
		\item milk
	\end{itemize}
	\item[o] Here is a changed example pay bills
	\item [$\heartsuit$] Here is a changed example do laundry
	\item [(a)] Here is a changed example using a literal (a)
	\item [OK] Here is a changed example using the word `OK'

\end{itemize}

\end{frame}

\section{Tabular and Tabbing}

\subsection{Tabbing}

\begin{frame}
\frametitle{Tabbing environment}

\begin{example}  
\begin{tabbing}
~a. 	[lup] ~~~\=	wolf.M.sg  ~~~~~~\= [lup\textsuperscript{j}]  ~~~\= 	`wolf-M.pl'\\%(\textit{lupi})	
~b. 	[alb] 	\> white.M.sg \> [alb\textsuperscript{j}] \> `white-M.pl' \\%(\textit{albi})
~c. 	[sar]	\> jump.1.sg \> [sar\textsuperscript{j}]	 \> `jump-2.sg'\\ %(\textit{sari})
\end{tabbing}
\end{example}

\end{frame}

\subsection{Tabular}

\begin{frame}
\frametitle{Tabular}


\begin{example}\label{morphologicalprimary}
\begin{tabular}{llll}
a.& rak & \ipa{ratS}$^j$ \\
& \textit{`crawfish'} & \textit{`crawfishes'}\\
b. & pas & \ipa{paS}$^j$\\
&\textit{`step'} & \textit{`steps-pl'}\\
\end{tabular}
\end{example}

\end{frame}

\section{Graphics}

\begin{frame}
\frametitle{Articulatory Description (Keating \& Lahiri 1993)}
\fbox{\includegraphics[height=2cm]{keating,lahiri-1993palate.eps}}\begin{tabular}{l}\vspace{-.8in}
~\\
1. corner of the alveolar ridge\\
2. diagonal\\
3. roof\\
4. soft palate
\end{tabular}

\end{frame}


\begin{example}Surface Inventory for Vowels\\
  \begin{vowel}[t]
    \putcvowel{i}{1}
\putcvowel{e}{2}
%    \putcvowel{\textipa{E}}{3}
    \putcvowel{a}{5}
\putcvowel{\textipa{O}}{6}
    \putcvowel{o}{7}
\putcvowel{u}{8}
    \putcvowel{\textipa{@}}{11}
    \putcvowel{\textipa{I}}{13}
\putcvowel{\textipa{U}}{14}
  \end{vowel} 
\end{example}


\section*{References}

\begin{frame}[shrink=5]
\frametitle{Selected References}
\begin{itemize}
\item Chitoran, I. 2001. The Phonology of Romanian: A Constraint-based Approach. New York, Mouton de Gruyter.
\item Kochetov, A. 2002. Production, Perception, and Emergent Phonotactic Patterns: A Case of Contrastive Palatalization.
New York, Routledge.
\item Keating, P. \& A. Lahiri. (1993) Fronted velars, palatalized velars, and palatals. Phonetica 50, 73-101.
\item Spinu, L. 2006. Perceptual properties of palatalization in Romanian. Paper presented at the 36th Linguistics
Symposium on Romance Languages, Rutgers University.
\item Stevens, K. 1989. On the quantal nature of speech, Journal of 
Phonetics 17, 3-46.
\item Stevens, K. 2003. Acoustic and Perceptual Evidence for Universal Phonological Features. In Proceedings of the 15th
ICPhS Barcelona, 33-38.
\end{itemize}

\end{frame}

\begin{frame}
~~~~~~~~~~~~~~~~~~~~Thank you!
\end{frame}
\end{document}


