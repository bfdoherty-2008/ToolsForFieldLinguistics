%this is a copy of ``new.tex'' i made around may 2003 while learnign tex. this doc was made nov 15 2003
\documentclass[12pt,twoside,leqno]{article}%leqno is to ahve hte numbers on the left for equations
%preamble is here, put commands that influence the 
%\usepackage{thisisapackagename}
\usepackage{syntonly}
%\syntonly %if this is uncommented this command to make the tex go faster it just checks syntax not make a new dvi file.

\usepackage[dvips]{graphicx} % this is to include graphics. save the file as a eps, or ps in gimp then type
%\includegraphics{filename} (uncommented in the body of the ducument)

\usepackage{verbatim} %inorder to include asii text files \verbatiminput{filename}

\hyphenation{so-lil-o-quy}%shows how an unknown word can be hyphenated throughout the document.
\begin{document}
\pagenumbering{roman}
\pagestyle{headings}
%\pagestyle{myheadings}
xdvi %\markboth{left-hand (even)page header}{Right-hand (odd) page header}
%in the book format because its default is doublesided printing you can have different headigns for the right and left hand pages. use \markboth to do this. otherwise you can just use \markright

\pagenumbering{arabic}

\title{MyFirst Document}
\author{Gina Cook \and Mrs. Fields\thanks{Without whose help, this endeavor would be fruitless}}
\date{May 25th 2003}%if the \date command is omitted the current date is used or the command \today can be used (anywhere in the document)

\maketitle
\tableofcontents
\clearpage %this makes the text go on the next page
\section{HERE I GO!}
\label{I_Go}
\subsection{subsection}
\subsubsection{subsubsection}
\paragraph{paragraph}
\subparagraph{subparagraph}

This is the text of my first document in \LaTeX!
A     second     sentance.
It's much more fun to do \LaTeXe{}than to
just read about it.
if you have an abreviathion that is a lowercase letter folowed by a period, use the space character (backslash) so that latex doesnt interprit it as a sentance boundary and put two spaces. this  etc.\ this should show up with only one space after it. etc. should show up with two spaces after it. upper case letters before a period are not treated as ends of sentances so you have to put a backslash folowwed by an at symbol. this is an examplE.\@ This is an example without the symbolS. ther should be only one space.. 
\section{Spacing}
Here begins the second paragraph. It follows a blank line. I don't have to indent; \LaTeX\ does that automatically.

\noindent This paragraph won't be indented, since it starts with a special command fo r stopping pharagraph indentation. 
\section{Qotes}
``this is a quote''

Yesterday Sally asked, ``Is this how to do quotes?'' She recalled, ``The book says: `there is a difference between open and close quote marks'\,''.

the command \, makes a small horizontal space\footnote{this was put between the single quote and the double quote}
\part{Part here doesnt change numbering}
\section{dashes}
Inter-word dashes, or hyphens, are made with one dash. Dashes for number ranges, like 6--12, are made with two dashes, and punctuation dashes are made with three dashes---like this!!!
\section{Symbols}Let's see what's behind door \#1---and it is---\$10,000!
its $-30\,^{\circ}\mathrm{C}$.
\# \$ \% \& \~{} $\sim$ \_ \^ \{ \} $<$ $>$ $\backslash$ \verb!\\! the double slashes make a new line but without a paragraph indentation.\verb!\\*! makes a pagebreak after the newline.
\section{Footnotes}
Cats\footnote[5]{you can specify the number or symbol for the footnote optional arguments must be placed just after the command and before the braces. Warm, soft, furry animals.} are very intelligent.
\section{I am considerate \protect\footnote{and protect my footnote}}
footnotes in headings need to be protected
\section{Margin Notes}
This is some text in a paragraph. It will have a marginal note\normalmarginpar \marginpar{this is a pretty note} which will be placed next to the paragraph.printing.\reversemarginpar\marginpar{this note will come out on the opposite side of the default note} the side taht the margin note will apear on depends on whether the is single or double sided printed

x=y+2

$x=y+2$
\section {My First Math Formula}
Here comes an in-line formula: $x^2+y^2=z^2$
if $x=1$ and $y=2$ then $z=3$:
Also, $x'+y' =z_{21}$.

\section{Comments}
%this is a comment which cannot be seen

This is real text. %This icoment won't print
%This whole line is a comment tha t won't print
Ths is more real text.

You can use the comment environment provided by the verbatim package. type
\begin{verbatim}
\usepackage{verbatim} in the preamble (between /documentclass and /begin{document}

then you can use
\begin{comment}
this wont be displayed.
\end{comment}
\end{verbatim}
This wont work inside the math environment (or other complicated environments).

%arguments go into the list of figures or list of tables\caption
%arguments goes into the table of contents list of figures or list of tables\addcontentsline
\section{Verbatim}
Here is some text which cdescribes the \verb!\footnote!command.The exclaimationpoints in the comand can be any symbol if you ruesing ! in your code use something else like a ?.

\verb!this wont print with symbols \latex\ !

\begin{verbatim}
I want all this text
to come out verbatim!!
Here is a \LaTeX\ command: \footnote(It won't work)
%This line will not be a comment.

\end{verbatim}
\section{Typestyles and Sizes}
Typestyle-changing commands give one a \bf powerful \rm feeling.

The \bf rain \rm in \it Spain \rm falls \sl mainly \rm on the \sf plain! \rm

\tiny this is tiny\\
\scriptsize THis is script size\\
\footnotesize This is footnote size\\
\small THis is small\\
\normalsize This is normal size\\
\large This is large\\
\Large This is larger\\
\LARGE This is very large\\
\huge THis is huge\\
\Huge This is very huge\normalsize\\

%to change the typestyle and size at the same time put the typestyle command after the size command

The \tiny frightened kitten \normalsize raised its fur and \it hissed \rm at the \Large big German Shepherd. \normalsize

The {\tiny frightened kitten} raised its fur and {\it hissed} at the {\Large big German Shepherd.}

\emph{if you use emphasizing inside a peice of emphasized text, then \LaTeX{} uses the \emph{normal} font for emphasizing.}
\section*{This Section will have no number and serves as a heading}

you must always tex a document twice inorder to get a correct table of contents.
\section[Entry for the T.O.C.]{Section Heading In the Text}
you can have an optional argument if you want the table of contents entry to be different from the section heading
\section*{this is the Heading}
\addcontentsline{toc}{section}{This is the Heading}
the addcontentsline must appear on the same page as the unnumberd heading inorder to have the right page number in the table of contents. this heading will appear in the table of contents dispite having no number.

\section[HEading for the TOC]{Heading for the Text:alternateing TOC and Text headings}
This is a section whose TOC entry is different from its heading in the text!

Chapter \ref{I_Go} contains some of my very first \LaTeX\ commands!
If latex isnt run twice then the references will apear as ?? it takes two runs for latex references to be resolved correctly. if you still have a ?? then the lable and the ref don't match.

to reference a  page use \verb!\pageref!  see Chapter \ref{I_Go} on page \pageref{I_Go} for more.(this is just an example)
\section{Hyphenation}
latex automatically hohenates words so if you ahve a word that isnt in its dictionary youc an specify its hyphatin in between the \verb!\documentstyle! and the 
\verb!\begin{document}! command.

\section{Quote vs Quotations}
For a shore quottion, or a series of one liners:

\begin{quote}
``Is this how to do quotes?'', asked Sallly.

``Very good, Sally!'', said Jane.

``The book says `there is a difference between open and close quote marks'\,'', said Sally.

``That is true.'', said Jane.
\end{quote}

For a quote of a paraghraph or more:

\begin{quotation}
Singing is perhaps the most personal of all the performing arts. WHne performing, a good singer makes each member of the audience feel as if he or she is receiving a generous and loving gift. Only in singing is ther ea direct, unhindered path between the muxic in the performer's heart and the listener's senses. 
\end{quotation}


\section{Making Lists}
Her comes a beautiful list:
\begin{itemize}
\item This is the first list item! Let's make it a long one so taht we can demonstrage how nicely \LaTeX\ indents list items.
\item[o] groceries
  \begin{itemize}
  \item potatoes
    \begin{itemize}
    \item red
  	\begin{itemize}
	\item russet
	\end{itemize}
    \end{itemize}
  \item celery
  \item frying chicken
  \item milk
  \end{itemize}
\item [$\heartsuit$] laundry
\end{itemize}

There are only four levels of list availlible.

\begin{enumerate}
\item THis is hte first level

  \begin{enumerate}
  \item This is hte second level

    \begin{enumerate}
      \item This is the third level

	\begin{enumerate}
	  \item This is the fourth level
	    \end{enumerate}
	\end{enumerate}
    \item This is the seecond item in the second level
      \end{enumerate}
  \item This is the second item in the first level
  \end{enumerate}
You can have an itemize inside of an enumerate and vice versa

\begin{enumerate}
\item \label{list:one} this is list item number one.
\item \label{list:two} Here is number two.
\item \label{list:three} And here is number three. 
\end{enumerate}

See Item \ref{list:two} in the list above for a description of number two.


To refer to a bibliography item in the style of a footnote, use the math superscript command, as in Keating$^{\ref{bib:keating}}$. Or you can just do it like this: see Reference~\ref{bib:keating} for a great story about underspecification.

The tilda in the code \verb!~! prints as a space and prevents a line break from occuring inbetween the units.

\section[References]{References}
\begin{description}
\item  \label{bib:barbosa} Barbosa, Pliínio Almeida  1996, ``At Least Two Macrorythmic Units are Necessary for Modeling Brazilian Portugues duration,'' {\it Proceedings of the First ESCA Tutorial Research Workshop on Speech Production Modeling and Fourth Speech Production Seminar} pp. 85-88,\\
http://www.lafape.iel.unicamp.br/Publica\%C3\%A7\%C3\%B5es/Autrans96.pdf.

\item \label{bib:boersma} Boersma, Paul \& David Weenink 2003, {\it Praat: Doing Phonetics by Computer.}  Version 4.0.43, http://www.praat.org.
\item \label{bib:cox} Cox, Felicity  2002, {\it The Acoustics of Speech},\\
 http://www.ling.mq.edu.au/units/sph301/consonants/nasalweb.html.
\item \label{bib:delvaux}Delvaux, Veronique, Alain Soquet \& John Kingston June 2002, {\it Production and Perception of French Nasal Vowels},\\
 http://www.ling.yale.edu:16080/labphon8/Poster\_Abstracts/Delvaux.html.
\item \label{bib:guion} Guion, Susan 2002, {\it Nasal, Approximate and Lateral Acoustics},\\
http://www.uoregon.edu/\~{}guion/411notes/nasacous.htm.
\item \label{bib:keating} Keating, Patricia A. 1988, ``Underspecification in phonetics,'' {\it Phonology 5.2}, pp. 275-292.
\item \label{bib:ohala} Ohala, John J. Draft 2001,``Aerodynamic Principles'' (Chapter 2), `` Acoustics'', (Chapter 3) {\it Phonology in Your Ear}, pp. 3-56.
\item \label{bib:ohalas} Ohala, John J.\& Manjari Ohala 1995, ``Speech perception and lexical representation of vowel nasalization in Hindi and English'', {\it  Phonology and Phonetic Evidence Papers in Laboratory Phonology IV}, Cambridge University Press, pp. 41-60.
\item \label{bib:zang} Zang, Jie 2002 {\it Nasalized Vowels},\\
 http://icg.harvard.edu/~ling113/lectures.
\end{description}

\section{Descriptive lists}
\begin{description}
\item [Dogs] Dogs, with their friendly obedient nature, make excellent pets. There are many differnt sizes of dogs, ranging from a bundle you can old in one hand to a 50--60 pound animal tha tbegins to resemle a horse. 
\item [Cat tfdtrfcs] Cats are ideal pets for people who are on-the-go. Independent and intellegent in nature, they do not require a great deal of attention. While being well able to entertain and take care of themselves, ats also offer warmth and affection to their owners.
\item [Birds] Birds add a splash of colour and a pleasant background music to the household. The patient bird owner can train his pet to talk and sit on his finger, and eve ride around town on his shoulder.
\end{description}
\section{Equations}
Here is a paragraph of text. It leats up to a brilliant (numbered) euation:
\begin{equation}
x=y+z
\label{eq:first}
\end{equation}

That was Equation~\ref{eq:first}.  (Number provided by the ref command.) And another paragraph may follow the euation. TO produce the same euation with out a number, type the following:
\begin{displaymath}
x=y+z
\end{displaymath}

Using the shorthand notation, \[x=y+z\] let us see what \LaTeX\ will do with a displayed equation entered in teh middle of a paragraph. 

\begin{equation}
\int_{0}^{\infty} f(x)=g(x)
\end{equation}

\[
\sum_{1}^{5}x=15
\]
\LaTeX\ Can have the sums and integrals taller or shorter depending on use of hte displaystyle of textstyle. Here is an in-line integral: $\int_{0}^{1}f(x)=g(x)$, and here is an in-line summation: $\sum_{1}^{5} x =15$. They look different from the displayed forms.

Here is a bunch of text to make a paragraph to see how the tall integral will look. Here is a {\it tall} in-line integral: $\displaystyle \int_{0}^{1}f(x)=g(x)$. And here is a {\it short} displayed summation:

\[
\textstyle \sum_{1}^ {5} x=15
\]
\begin{equation}
x=i\scriptstyle j \scriptscriptstyle k
\end{equation}
\\
\`{o}
\'{O}
\^{o}
\"{O}
\~{o}
\={O}
\.{o}
\u{O}
\v{o}
\H{O}
\t{ts}
\c{O}
\d{o}
\b{O}
\oe
\OE
\ae
\AE
\aa
\AA
\o
\O
\l
\L
\ss
?`
!`
\copyright
\pounds
\P75
\S
\dag
\ddag

?`Como est\'{a} ustead?\\
Notre amour est chose l\'{e}g\'{e}re.\\
Ein V\"{o}gelein fliegt \"{u}ber den Rhein.\\

Calling the pavgenumbering command in teh middle of hte documetn has the pages start over again. the style can also be changed for example the table of contents can be in roman and the rest of the document can be in arabic. and the appendix can be in capital Roman letters etc...
\section{Array Example}
\[
\begin{array}{ccccc}
x_1 & 0 & 0 &\cdots & 0 \\
0   & x_2 & 0 & \cdots & \vdots \\
0 & 0 & \ddots & 0 & \vdots \\
\vdots & \vdots & 0 & \ddots & 0 \\
0 & \cdots & \cdots & 0 & x_n \\
\end{array}
\]

Here is an example using $\alpha$ and $\beta$. $\cal G$ is also used.
\begin{equation}
{\cal G } = \frac{\alpha +\beta}{\alpha - \beta}
\end{equation}

to keep normal text in a document use mbox

\begin{equation}
f(x)=a \mbox{ for negative values of } x
\end{equation}
\begin{equation}
f(x)=a for negative values of 
x
\end{equation}

You can also use mbox to keep groups of text together across a line break
\begin{verbatim}
my phone number \mbox{0116 291 2319}
\end{verbatim}
$\vec{P}$\\
\[
\overbrace{\overbrace{a} + b} + \underbrace {\underbrace{c} + d}
\]

\[
X\stackrel{a}{\rightarrow} Y \stackrel{\textsyle b}{\rightarrow} Z
\]
\section{Simple Tabbing}
\begin{tabbing}
This is a line of text with a tab stop \= here \\
Second line: go to tab \>here see? \\
Now set new tabs: \= one and \=two and \=three \\
Go to \>one \>two \>three\\
Any line ending with kill will not be printed\\
this will not print \kill
\end{tabbing}

Unless the tabular environment is surrounded by blank lines it will be treated as a letter and surrounded by text
\begin{tabular}{|lr|cp{.5in}|}\hline
\multicolumn{4}{c}{Overall heading of Table}\\\hline
\multicolumn{2}{c}{Column 1 \& 2}& \multicolumn{2}{c}{Column 3 \& 4}\\\hline
left justified & right justified & centered & right justified\\[.25in]\hline\hline
\multicolumn{1}{c}{one} & two & three & four\\ \cline{2-4}%multicolumn can be used to overide the justification of the column
1&2&3&4\\
i&ii&iii&iv
\end{tabular}like this one is.Note, if the first character in a line is [ you have to put \verb!{}[! so that latex doesnt think its a measurement argument.

\begin{center}
\begin{tabular}{|l|r@{.}l|}\hline
\multicolumn{3}{|c|}{\bf Household Budget}\\\hline\hline & \multicolumn{2}{c|}{\bf \% of }\\
\bf Item & \multicolumn{2}{c|}{\bf Budget}\\ \hline
Housing & 50 & 0 \\
Food & 25 & 0 \\
Toys & 10 & 5 \\
Pet Supplies & 7& 5\\
Clothes & 5 & 25\\
Charity & 1 & 75 \\ \hline
\end{tabular}
\end{center}

The numbers are centerd around the decimal points that is created with the r- column.

A top-aligned table:
\begin{tabular}[t]{cc}
a & b \\
c & d
\end{tabular}
and a bottom-aligned table:
\begin{tabular}[b]{cc}
a & b \\
c & d 
\end{tabular}
\subsection{use of equarray for multi-line formulas}
Arrays can only be used in math mode.

here is a formula contaioning equations~\ref{line1} and~\ref{line3}.
%\begin{equation} eqnarray automatically pyuts you into math mode, eqnarray is a special array for thest types of arrays it als moakes a number on the side for each equation.
\begin{eqnarray}
x& \leq &y+z+p \label{line1}\\
y-y'&=&x-z \nonumber\\
z&=&x-y \label{line3}
\end{eqnarray}
%\end{array}
%\end{equation}

you can use eqnarray* to not have any equation numbers in the array

\fbox{text in a box}.

We are now on Page \thepage\ of Section \thesection\ of Chapter \thechapter.

\renewcommand{\thefigure}{\arabic{chapter}.\Alph{figure}}
\setcounter{chapter}{2}
\setcounter{figure}{3}
\begin{figure}[ht]%the h=here t=top b=bottom p=floater page
\vspace{.5in}
\caption{This is a sample figure with a customized number including a period.}
\end{figure}
\raisebox{0pt}[0pt][0pt]{\large%
\textbf{Aaaa\raisebox{-0.3ex}{a}%
\raisebox{-0.7ex}{aa}%
\raisebox{-1.2ex}{r}%
\raisebox{-2.2ex}{g}%
\raisebox{-4.5ex}{h}}}
he shouted but not even the next one in line noticed that something terrible had happened to him.
\appendix
\section{THIS IS AN APPENDIX}
THis is the text of the Appendix.
\subsection {Subheading In An Appendix}
How does this look?
\end{document}
